\documentclass[a4paper,10pt]{article}
\usepackage[utf8]{inputenc}
\usepackage{listings}
\usepackage{graphicx} 
\usepackage{titlesec}

%opening
\title{Résolution numérique en Python}
\author{WANG Jinxin 3404759}

\begin{document}

\maketitle

\begin{abstract}
\indent
1) Méthode de résolution numérique dans les systèmes dynamiques\\
\indent
2) Équation de diffusion\\
\end{abstract}

\newpage
\section{Méthode de résolution numérique dans les systèmes dynamiques}

\subsection{ Compléter (3) afin que le code implémente la méthode d'Euler }
\begin{lstlisting}[language=Python]
    for i in range(n-1):
	x[i+1] = x[i] + h * func(x[i]) # (3)
\end{lstlisting} 

\subsection {En compléter (1) et (2), résoudre numériquement l'équation de croissance logistique}
\begin{lstlisting}[language=Python]

    def f(x):
	return 0.1 * x * ( 30 - x ) # (1)

    def init_x(val_init_x, n):
	if type(val_init_x) == type([]):
	    x = np.zeros([n,len(val_init_x)])
	else:
	    x = np.zeros([n,])
	return x
    
    def euler(func, val_init_x, tf = 100, n = 500):
	h = tf/float(n)
	x = init_x(val_init_x, n)
	x[0] = val_init_x # (2)
	for i in range(n-1):
	    x[i+1] = x[i] + h * func(x[i])
	return x
	
    euler(f, 5, tf=10)
    euler(f,50, tf=10)   

\end{lstlisting} 
\subsection{ DM Tracer sur un meme graphe l'équation de la popluation poiur x0=5 et x0=50 }
\begin{center}
\includegraphics[height=2.5in]{TD2-Q3.png}
\end{center}

\subsection { Runge-Kutta }
\begin{lstlisting}[language=Python]

    def RK2(func, val_init_x, tf = 100, n = 500):
	h = tf/float(n)
	x = init_x(val_init_x, n)
	x[0] = val_init_x
	for i in range(n-1):
	    x[i+1] = x[i] + (h/2) * func(x[i]) 
		      + (h/2) * func(x[i]+h*func(x[i]))
	return x

    RK2(f, 5, tf=10)

\end{lstlisting} 

\begin{center}
\includegraphics[height=2.5in]{TD2-Q5.png}
\end{center}

\subsection { Un programme qui évqlue le moyenne du carré de l'erreur }
\begin{lstlisting}[language=Python]

    def calcul_erreur_Q6(func, val_init_x, tf = 100, n = 500):
	return ((sum(euler(func, val_init_x, tf, n)) 
		  - sum(RK2(func, val_init_x, tf, n)))**2)/n

\end{lstlisting}

\subsection { En utilisant la méthode d'Euler, modifier le code pour résoudre le système}
\begin{lstlisting}[language=Python]

    def f2(variable):
	[ x, y ] = variable
	return np.array([0.25*x - 0.01*x*y, 0.01*x*y - y])

    def init_x(val_init_x, n):
	if type(val_init_x) == type([]):
	    x = np.zeros([n,len(val_init_x)])
	else:
	    x = np.zeros([n,])
	return x    
	
    euler(f2,[80,30])

\end{lstlisting}

\subsection { Tracer le portrait de phase pour x(0) = 80 et y(0) = 30 }

\begin{center}
\includegraphics[height=2.7in]{TD2-Q8.png}
\end{center}

\begin{lstlisting}[language=Python]
    points = RK2(f2,[80,30])
    plt.plot(points[:,0], points[:,1], label='systeme 1, x0=80, y0=30 RK2')
    points = euler(f2,[80,30])
    plt.plot(points[:,0], points[:,1], label='systeme 2, x0=80, y0=30 euler')
\end{lstlisting}

\subsection { La solution obtenue correspond-t-elle au résultat attendu }
    Le figure obtenu correspond au résultat attendu. En temps passe, la popluation circule et approximate. Ils sont en fin stables. 

\section{Équation de diffusion}
\subsection { simuler l'évolution de la diffusion }
\begin{lstlisting}[language=Python]
# ---- Exercise 2 Equation de diffusion ----
import numpy as np
import matplotlib.pyplot as plt

from mpl_toolkits.mplot3d import Axes3D
from matplotlib import cm
from matplotlib.ticker import LinearLocator, FormatStrFormatter

tf = 100; nt=100; nx=100; a=1e-3; x=10
dt = tf/float(nt)
dx = x/float(nx) 
x0 = nx/2

U  = np.zeros([nt,nx])
U[0, x0] = 1

for k in range(0, nt-1):
    for i in range(1,nx-1):
        U[k+1,i]= U[k,i]+a*U[k,i+1]-2*U[k,i]+U[k,i-1] # (4)

fig = plt.figure()
ax  = fig.gca(projection ='3d')
X   = np.arange(0, x, dx); T = np.arange(0,tf,dt)
X, T= np.meshgrid(X,T)
surf= ax.plot_surface(X, T, U, rstride=1, cstride=1, cmap=cm.coolwarm, linewidth = 0, antialiased =False)
fig.colorbar(surf)
plt.show()
\end{lstlisting}

\subsection{La condition de bord dans le programme précedent correspond à 1}

\end{document}
